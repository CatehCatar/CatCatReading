
\chapter{Пропаганда и прогнозы}
Русско-японской войне предшествовало продвижение Российской Империи на Дальнем Востоке. Далеко не все конфликты удавалось решать миром. Напряжение витало в воздухе, а иногда, как в дни боксерского восстания, перерастало в открытые столкновения.

В книге Ирины Сергеевны Рыбачонок «Закат великой державы. Внешняя политика России на рубеже XIX‒XX вв.: цели, задачи и методы» приведена реакция отечественной прессы на дальневосточные дела.

Либеральная пресса кричала о некой самобытности Китая, которую потревожили колонизаторы-европейцы. Восстание Ихэтуаней казалось случайностью, недоразумением. Публицисты считали, что после кризиса древняя империя погрузится обратно в многовековой сон.

Наивными и поверхностными выглядят сейчас эти домыслы. Интересно сравнить их с консервативной прессой того времени.

«Китай окончательно проснулся и стал, как непримиримый враг, лицом к лицу ко всем своим соседям. Одних он боится, других презирает, но всех одинаково ненавидит. На историческую сцену выступили для единоборства монгольская и христианская культура, Европа и Азия, и единоборство это, очевидно,продолжается не год, не два, а с большими или меньшими перерывами целые десятилетия, а может быть, и сотни лет».

Это отрывок из статьи «Мировое значение китайского вопроса», написанной В. А. Гринмутом, редактором консервативной газеты «Московские ведомости».

В том же номере "Московских ведомостей" вышла статья философа В. С. Соловьева об опасности Китая. Владимира Сергеевича вряд ли можно отнести к консерваторам или монархистам: круг его идей, связанных с христианской мыслью, столь велик, что вчертить его в квадрат политических координат или линию «левые – правые» невозможно. Но именно он ввел в оборот термин панмонголизм, опасаясь нашествия с Востока, и даже написал стих об этом. И. С. Рыбачонок указывает, что Соловьев опирался на суждения своего отца, который показывал смену ведущих народов в истории.

Опасения по поводу Китая высказывались и позднее. Скандальный реакционер-черносотенец Пуришкевич гастролировал по городам с лекциями «Проснувшийся Китай как угроза русскому переселенческому движению». В них он утверждал: «<…> китайцы с алчностью подглядывают на земельные богатства Сибири и Дальнего Востока. Кроме того, у китайцев с нами старые счета. Амурская область ещё недавно составляла достояние китайцев, и теперь они мечтают о том, чтобы возвратить это достояние. [...] Через три года Китай будет располагать 30 дивизиями, а через семь лет – 52-мя. Т[о] е[сть] против нас окажется миллионная китайская армия».

Конечно, мотив консерваторов можно понять: Россия продвигалась на Дальнем Востоке, КВЖД представляло государство в государстве, Маньчжурию колонизировали. Такие алармистские вопли были всего лишь обоснованием захватнической политики. Но они оказались в итоге более точными. Либеральные рассуждения о «вековом сне» представляли же наивный европоцентричный стереотип.

Китай скинул всех захватчиков, вышел из гражданской войны с марксизмом – пусть и в сталинском изводе – на знамени, прошел Культурную революцию, а потом успешно скрестил авторитаризм и рыночную экономику.

И даже если Китай не справится с вызовами вроде демографической ямы или торговой войны с США его успехи, даже если китайская угроза преувеличена, его гигантские проекты, такие как телескоп FAST или AliExpress все равно будут впечатлять.

Автор Виктор Пепелов. Оригинал \url{https://vk.com/wall-162479647_99435}

\#Пепел@catx2

\#Заметка@catx2
