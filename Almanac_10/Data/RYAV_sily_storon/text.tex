
\chapter{Была ли Россия отсталой?}

Когда в Российской Империи появились первые пароходы, в Японии царил сёгунат и голод. Когда Россия приступила к созданию броненосного флота, то в Японии только разворачивалась революция Мэйдзи. Почему же Россию считают отсталой? Была ли война проиграна до начала или нет?

Русские создавали флот с нуля дважды. Первый раз – при Петре I, второй раз – в конце XIX в. В первый раз у них было два направления для экспансии и два грозных противника. Во второй раз – соперничество с самыми крупными морскими державами, а еще оборона двух морей и одного океана. Поэтому при оценке сил двух империй речь пойдет, в первую очередь, о кораблях. 

\section{Русский флот: от Крыма до Японии.}

В советской историографии красной нитью шла речь об «отсталости Царской России». При Сталине отношение к отечественной истории отошло от привычного охаивания в духе школы Покровского. Но вот Русско-японскую войну всегда выставляли как образец позора Российской империи и Романовых. После СССР появилась прямо противоположная точка зрения. Так где же истина?

Вторая половина XIX в. – это время индустриализации военного комплекса. Крымская война была последним крупным конфликтом с применением парусного флота (2). После нее стало ясно: главную роль играют технологии.

\begin{figure}[h!tb] 
	\centering\includegraphics[scale=0.3]{Data/RYAV_sily_storon/rO3nWL7-ijc.jpg}
	%	\label{fig:scipion} % Unique label used for referencing the figure in-text\end{document}
	%	%\addcontentsline{toc}{figure}{Figure \ref{fig:placeholder}} % Uncomment to add the figure to the table of contents%----------------------------------------------------------------------------------------
	\caption{Бой русского пароходофрегата «Владимир» и турецкого парохода «Перваз-Бахри». Художник А. Боголюбов. Это небольшой бой времен Крымской войны стало первым сражением пароходов в истории человечества.}%	CHAPTER 2
\end{figure}

В конце XIX в. вплоть до начала Первой мировой войны в Европейских державах настал период интенсивного военно-промышленного взаимодействия. Военный историк Мак-Нил датирует его с 1884 по 1914 г. На смену фрегатам с белоснежными парусами пришли огромные плавучие броненосцы из стали, проворные катера и подлодки. Торпеды, радио, системы управления огнем стали залогом победы в морских баталиях.

Технологии развивались стремительно. Стратегическая мысль просто не успевала за ним. Прежде адмиралы и министры понимали с ходу каждую инновацию. Теперь же на диалог с изобретателями уходили дни, а на испытания новинок недели и месяцы учений. «Математическая затруднительность проблемы, совершенно явственно превосходившая уровень знаний большинства людей даже из самого узкого круга посвященных, лишала политику минимальной рациональности», – так описывает Мак Нил ситуацию перед Первой Мировой Войной. Эти слова верны и для предшествующих десятилетий.




Назвать Россию отсталой страной сложно хотя бы из-за того, что модернизация флота началась сразу после Крымской Войны. Но и впадать в идеализацию не стоит: завязанная на государство экономика, крепостничество, отсутствие многих отраслей стали препятствиями для создания стальных кораблей. (3)
Не лишним при описании истории Русского флота будет и слово «Геополитика». Кто сильнее: Слон или Кит? Все зависит от среды. Россия не была первой на море, но показала себя неплохой на суше, потопив Наполеона в бескрайних пространствах Восточно-европейской равнины и подплывая через Хартленд Средней Азии к Индии.


Поэтому после Крымской войны о лидерстве на море речь не шла.
\begin{textcitation}
{	
		«Смирившись с невозможностью создать в ближайшее время боевой потенциал флота, равный боевым потенциалам флотов Англии и Франции, и в то же время не желая терять престижа России как одной из крупных морских держав, царское правительство при определении программы нового судостроения исходило из того, что «Россия должна быть первоклассною морскою державою, занимать в Европе третье место по силе флота после Англии и Франции и должна быть сильнее союза второстепенных морских держав» (имелись в виду Пруссия, Швеция и Дания. — Авт.)»}
\end{textcitation} 
пишут В. А. Золотарев и И. А. Козлов в«Трех столетиях Российского флота».\url{http://militera.lib.ru/h/zolotarev_kozlov2/11.html}

\begin{figure}[h!tb] 
	\centering\includegraphics[scale=0.2]{Data/RYAV_sily_storon/jYR7YIwi8uE.jpg}
	%	\label{fig:scipion} % Unique label used for referencing the figure in-text\end{document}
	%	%\addcontentsline{toc}{figure}{Figure \ref{fig:placeholder}} % Uncomment to add the figure to the table of contents%----------------------------------------------------------------------------------------
	\caption{Первый русский броненосец Пётр Великий. Изначально относился к классу Мониторов. Фотографии взяты с сайта «Цусима» \url{http://tsushima.su/petrvelphotoru/} Фото 1}%	CHAPTER 2
\end{figure}


\begin{figure}[h!tb] 
	\centering\includegraphics[scale=0.2]{Data/RYAV_sily_storon/wiYS26wh4ZQ.jpg}
	%	\label{fig:scipion} % Unique label used for referencing the figure in-text\end{document}
	%	%\addcontentsline{toc}{figure}{Figure \ref{fig:placeholder}} % Uncomment to add the figure to the table of contents%----------------------------------------------------------------------------------------
	\caption{Первый русский броненосец Пётр Великий. Изначально относился к классу Мониторов. Фотографии взяты с сайта «Цусима» \url{http://tsushima.su/petrvelphotoru/} Фото 2}%	CHAPTER 2
\end{figure}
\begin{figure}[h!tb] 
	\centering\includegraphics[scale=0.2]{Data/RYAV_sily_storon/dQQjFe6_e2U.jpg}
	%	\label{fig:scipion} % Unique label used for referencing the figure in-text\end{document}
	%	%\addcontentsline{toc}{figure}{Figure \ref{fig:placeholder}} % Uncomment to add the figure to the table of contents%----------------------------------------------------------------------------------------
	\caption{Первый русский броненосец Пётр Великий. Изначально относился к классу Мониторов. Фотографии взяты с сайта «Цусима» \url{http://tsushima.su/petrvelphotoru/} Фото 3}%	CHAPTER 2
\end{figure}

Тем не менее, за 15 лет после Парижского договора на Балтике появился броненосный флот, третий по силе в Европе. Со стапелей сошел и первый полноценный эскадренный броненосец, «Петр Великий». Затем на верфях создали и броненосные крейсеры, быстрые корабли с легкой броней, предназначенные для долгих автономных походов и действий на коммуникациях. Хуже обстояли дела на Черном Море. Южные рубежи решили поначалу охранять «Поповками», огромными круглыми бронированными плавучими крепостями. Они оказались тяжелыми, дорогими и абсолютно бесполезными. Одну такую «тарелку» сделали в Питере и собрали в Николаеве, другую смогли уже построить целиком там. Не всегда развитие технологий идет гладко, как в компьютерной игре от “Парадоксов” или “Цивилизации”. Что и говорить о морской тактике и стратегии. Некоторые светлые умы предлагали даже оснащать корабли таранами и пробивать противника как в античные времена.

\begin{figure}[h!tb] 
	\centering\includegraphics[scale=0.4]{Data/RYAV_sily_storon/GJfhe1Pp85A.jpg}
	%	\label{fig:scipion} % Unique label used for referencing the figure in-text\end{document}
	%	%\addcontentsline{toc}{figure}{Figure \ref{fig:placeholder}} % Uncomment to add the figure to the table of contents%----------------------------------------------------------------------------------------
	\caption{Слабоопознанный плавающий объект «Поповка». Фотография взята с сайта «Поп-механика» \url{https://www.popmech.ru/weapon/13814-plavayushchie-tarelki-absurd/}}%	CHAPTER 2
\end{figure}

В 1880 был принят новый план строительства флота. Мы создавали уже не оборонительно-прибрежный, а океанский. Как видно из таблицы, упор был сделан на Балтику: царское правительство заслуженно опасалось Германской Империи. К тому же Петербург был в то время промышленным центром России.

\begin{figure}[h!tb] 
	\centering\includegraphics[scale=0.4]{Data/RYAV_sily_storon/eAX9dphs0L4.jpg}
	%	\label{fig:scipion} % Unique label used for referencing the figure in-text\end{document}
	%	%\addcontentsline{toc}{figure}{Figure \ref{fig:placeholder}} % Uncomment to add the figure to the table of contents%----------------------------------------------------------------------------------------
	\caption{Источник: Золотарев В.А., Козлов И.А. Три столетия Российского флота»}%	CHAPTER 2
\end{figure}

В это же время сложилась классификация кораблей флота. Расскажу о ней вкратце (цифры взяты из монографии В. А. Золотарева и И. А. Козлова «Три столетия Российского флота».
1) Эскадренный броненосец или просто броненосец. Огромная плавучая гора с кучей пушек. Минус – неповоротливость. Некоторые же из них, такие как Полтава, еще и не отличались дальностью хода.

\textbf{«Водоизмещение 10–15 тыс. т; вооружение: артиллерийское — четыре 305-мм, до двенадцати 152-мм, до двадцати 75-мм и до тридцати 47–37-мм орудий; торпедное — до четырех надводных и двух подводных торпедных аппаратов; бронирование 406–250 мм; скорость 17–18 узлов; дальность плавания до 8 тыс. миль».}


Первым Броненосцем, если не считать корабль «Петр Первый», был «Император Александр I». Но вплоть до русско-японской войны при постройке вот таких Левиафанов ориентировались на сделанный чуть позже, в 1891 г., «Наварин». В дальнейшем появилось немало броненосцев разных конструкций. Самой многочисленной серией был тип «Бородино».
\begin{textcitation}
«Создание кораблей типа «Бородино» явилось несомненным достижением российской промышленности. Более многочисленные серии броненосцев до этого строились только для британского флота, четыре серии по пять кораблей в каждой в 1895-1907 гг. были построены для флота Германии»,	
\end{textcitation}
подчеркнул Владимир Грибовский про эти суда.\url{http://tsushima.su/RU/libru/i/Page_6/page_15/grib-borodino/}

\begin{figure}[h!tb] 
	\centering\includegraphics[scale=0.2]{Data/RYAV_sily_storon/pDAKm_tQdIg.jpg}
	%	\label{fig:scipion} % Unique label used for referencing the figure in-text\end{document}
	%	%\addcontentsline{toc}{figure}{Figure \ref{fig:placeholder}} % Uncomment to add the figure to the table of contents%----------------------------------------------------------------------------------------
	\caption{Эскадренный броненосец "Бородино" на малом Кронштадтском рейде. Фотографии взяты с сайта «Цусима»
	}%	CHAPTER 2
\end{figure}


2) Крейсер 1-ого ранга. Проворная плавучая гора с броней, способная как сражаться в эскадре, так и становиться странствующим рыцарем морей, стальной хищной акулой. Самый известный из них – «Аврора» (тип Паллада).

\textbf{«Водоизмещение достигало 12 тыс. т, скорость — 20 узлов, дальность плавания — 8000 миль; вооружение: артиллерийское — четыре 203-мм, шестнадцать 152-мм, до тридцати 37-мм орудий, торпедное — до четырех надводных торпедных аппаратов; бронирование — до 203 мм».}

\begin{figure}[h!tb] 
	\centering\includegraphics[scale=0.5]{Data/RYAV_sily_storon/o2cXT_E28cA.jpg}
	%	\label{fig:scipion} % Unique label used for referencing the figure in-text\end{document}
	%	%\addcontentsline{toc}{figure}{Figure \ref{fig:placeholder}} % Uncomment to add the figure to the table of contents%----------------------------------------------------------------------------------------
	\caption{«Аврора». Фото с официального сайта \url{http://aurora.org.ru/info/krejser-1-go-ranga-avrora-na-vechnoj-stoyanke-u-petrovskoj-naberezhnoj-sankt-peterburg/}
	}%	CHAPTER 2
\end{figure}

3) Крейсер 2-ого ранга – небольшой корабль для разведки и обороны.

\textbf{«Водоизмещение от 3000 до 6000 т, скорость до 25 узлов, дальность плавания до 4000 миль; вооружение: артиллерийское — восемь 152-мм, двадцать четыре 75-мм, восемь 37-мм орудий; торпедное — до четырех торпедных аппаратов».}

\begin{figure}[h!tb] 
	\centering\includegraphics[scale=0.3]{Data/RYAV_sily_storon/f7G2P9E6i0c.jpg}
	%	\label{fig:scipion} % Unique label used for referencing the figure in-text\end{document}
	%	%\addcontentsline{toc}{figure}{Figure \ref{fig:placeholder}} % Uncomment to add the figure to the table of contents%----------------------------------------------------------------------------------------
	\caption{Крейсер II ранга «Новик» }
	%	CHAPTER 2
\end{figure}

4) Канонерская лодка. Небольшое судно, чтоб бить врага вблизи берега.

\textbf{«Водоизмещение до 1500 т, скорость до 15 узлов и по два орудия калибром от 152 до 225 мм».}

\begin{figure}[h!tb] 
	\centering\includegraphics[scale=0.3]{Data/RYAV_sily_storon/7qe4VyVHQn0.jpg}
	%	\label{fig:scipion} % Unique label used for referencing the figure in-text\end{document}
	%	%\addcontentsline{toc}{figure}{Figure \ref{fig:placeholder}} % Uncomment to add the figure to the table of contents%----------------------------------------------------------------------------------------
	\caption{Канонерская лодка «Кореец»
	 }
%	CHAPTER 2
\end{figure}

5) Миноносец и эскадренный миноносец. Маленький, но злой кораблик с торпедами.

\textbf{«Водоизмещение эскадренных миноносцев до 350 т, скорость до 27 узлов, одно 75-мм и пять 47-мм орудий, три торпедных аппарата; у миноносцев водоизмещение до 180 т, скорость до 24 уз., три 37-мм орудия, два торпедных аппарата».}

\begin{figure}[h!tb] 
	\centering\includegraphics[scale=0.4]{Data/RYAV_sily_storon/a1QM7sSGutc.jpg}
	%	\label{fig:scipion} % Unique label used for referencing the figure in-text\end{document}
	%	%\addcontentsline{toc}{figure}{Figure \ref{fig:placeholder}} % Uncomment to add the figure to the table of contents%----------------------------------------------------------------------------------------
	\caption{Миноносец «Взрыв»
	}
	%	CHAPTER 2
\end{figure}

6) Минный транспорт. Штука, ставящая мины.

\textbf{«Водоизмещение достигало 2800 т, скорость — до 17 узлов, вооружение состояло из пяти 75-мм и семи 47-мм орудий; могли принимать до 300–400 мин».}

\begin{figure}[h!tb] 
	\centering\includegraphics[scale=0.3]{Data/RYAV_sily_storon/w5qmReYSZTc.jpg}
	%	\label{fig:scipion} % Unique label used for referencing the figure in-text\end{document}
	%	%\addcontentsline{toc}{figure}{Figure \ref{fig:placeholder}} % Uncomment to add the figure to the table of contents%----------------------------------------------------------------------------------------
	\caption{Минный заградитель «Амур»
	}
	%	CHAPTER 2
\end{figure}

В 80-х гг начался новый этап строительства флота. В 90-е гг. русский изобретатель А. П. Давыдов создал систему автоматического управления огнем. В тоже время появилась никелированная сталь, радиосвязь и торпеды. «К началу русско-японской войны на вооружении [472] русского флота имелись 45-см торпеды, снабженные гироскопическим прибором управления движением торпеды по направлению, с дальностью хода 2000 м (при скорости 36 узлов) и 1000 м (при скорости 32 узла)».
С другой же стороны, немалая доля «начинки» для тех же кораблей была иностранного производства.
\begin{textcitation}
«При подготовке к походу на «Бородино» были установлены оптические прицелы системы Перепелкина для орудий калибром от 75 до 305 мм, два дальномера системы Барра и Струда, станция беспроволочного телеграфирования системы "Сляби-Арко" германской фирмы "Телефункен", стрелы Темперлея и устройства Спенсера-Миллера для погрузки угля»
\end{textcitation}
 пишет В. Ю. Грибовский в работе об эскадренном броненосце Бородино.

Но такая практика не была редкостью и на Западе. Например, итальянские крейсера типа «Джузеппе Гарибальди» оснащались британской артиллерией фирмы Армстронг. Японские корабли «Касуга» и «Нисима», сделанные на их основе, также несли английские пушки. Работ, подсчитывающих точное соотношение отечественных и западных устройств и оружий, а также технологий, я не нашел.
Благодаря военной реформе Милютина, флот комплектовался не через рекрутские наборы, а ВСЕСОСЛОВНУЮ повинность. Модернизировались и военно-морские училища.
Но почему далеко не худший флот мира проиграл стране, которая в 1850 гг. находилась в состоянии средневековья?

\section{Сравнение Русского и Японского флота}

В целом, русский флот превосходил японский. Но он оказывался распылен на трех направлениях: Тихий Океан, Балтика и Черное море. Проливы и северные воды тревожили Царское правительство, это становится ясно из прошлой статьи про дипломатию. Не выветрилась из памяти и русско-турецкая война 1877-1878 гг.
Но и Дальний восток не уходил из внимания чиновников. Программа по тихоокеанскому флоту была пересмотрена. Готовились резервные эскадры и специальная программа для Дальнего Востока.
\begin{textcitation}
 «Предусматривалось построить (сверх программы 1895 г.) 5 эскадренных броненосцев, 16 крейсеров, 2 минных заградителя и 36 эскадренных миноносцев и миноносцев. Выполнение этой программы должно было закончиться в 1905 г.»
\end{textcitation}
пишут В. А. Золотарев и И. А. Козлов. Именно в это время часть кораблей заказали за границей.
Это привело к тому, что “двуглавый орел” буквально метался с одного конца Евразии на другой. Одни высшие чиновники считали, что приоритет должен остаться за Востоком. Другие, что за Западом.

\begin{textcitation}
«В состав русской эскадры на Тихом океане входили 7 эскадренных броненосцев, 4 броненосных крейсера 1-го ранга, 5 бронепалубных крейсеров 1-го ранга, 2 крейсера 2-го ранга, 6 канонерских лодок, 25 эскадренных миноносцев, 10 миноносцев, 2 минных крейсера, 2 минных заградителя. 1 эскадренный броненосец, 2 крейсера 1-го ранга 1 крейсер 2-го ранга, 7 эскадренных миноносцев, 4 миноносца и 3 транспорта находились в пути на Дальний Восток под командованием вице-адмирала А.А. Вирениуса»
\end{textcitation}
 подытоживает О. Р. Айрапетов.

Против него встал японский флот. 6 новейших эскадренных броненосцев и 6 бронированных крейсеров, т.н. флот «6 на 6». Кроме того, у японцев имелось 6 броненосцев береговой обороны, 7 крейсеров 1-го ранга, 11 крейсеров 2-го ранга, 8 канонерских лодок, 4 минных крейсера и 47 миноносцев. Одновременно, из Средиземного моря в Японию шли 2 броненосных крейсера, купленные в Италии. Таким образом, японцы получили превосходство в ударных силах.

Превосходили ли японские корабли наши? Да.
\begin{textcitation}
 «Три русских броненосца — «Петропавловск», «Севастополь» и «Полтава» являлись уже устаревшими кораблями. <…>. Известный справочник Джейна за 1904 г. соотносил их боевую силу как 0,8 к 1,0 в пользу последних. Кроме того, машины «Севастополя», изготовленные Франко-Русским заводом в Петербурге, отличались низким качеством изготовления и сборки. Даже на официальных испытаниях в 1900 году «Севастополь» не смог развить контрактной скорости (16 узлов), а к началу военных действий с трудом развивал 14» \url{https://cmboat.ru/rusmin1/minonosec3/}
\end{textcitation}
пишет С. В. Несолёный в книге Миноносцы Первой эскадры флота Тихого океана в русско-японской войне (1904-1905 гг.)

И самый главный фактор заключался в том, что Императорский флот Японии превосходил русский в типизации. 
\begin{textcitation}
«Японские эскадренные броненосцы являлись однотипными кораблями новейшей постройки, тогда как русские эскадренные броненосцы, построенные по различным судостроительным программам с интервалом времени до семи лет, принадлежали к четырем различным типам кораблей, обладавшим различными тактико-техническими данными»
\end{textcitation}
продолжает С. В. Несоленый. Да, это не было каким-то сильным превосходством, какое было, например, у регулярной армии над ополчением. Но все равно это дало японцам преимущества. Например, выигрыш в скорости. Русские броненосцы шли медленней на два узла.

При этом японские корабли закупались на Западе. Все броненосцы и практически все броненосные крейсеры сходили на воду с верфей Британии. Большая часть русских судов создавалась – пусть и по западным лекалам – у нас.

И здесь мы отойдем от строго научной литературы в область публицистики. Блогер Половинкин Дмитрий Сергеевич (ЖЖ-юзер Олд-Адмирал) выполнил любительское (но очень качественное) исследование японского флота на основе нескольких научных монографий. Приведу два тезиса.

Японские броненосцы «Сикисима», «Хацусе», «Асахи» и «Микаса» не уступали по своим характеристикам крупнейшим английским кораблям. А вместе эта четверка могла бы набить морду и одному (!) из британских флотов.

Японские броненосные крейсера же имели перекос в сторону боевой мощи, жертвуя остальными параметрами. Такими же характеристиками обладали и вышеупомянутые итальянские корабли Касуга и Нисима. И страна Восходящего Солнца могла себе позволить, так как их противники находились неподалеку. Русские же верфи и промышленные центры располагались на другом конце земного шара. (!)

\begin{textcitation}
«Таким образом, как мы с вами могли убедиться, практически весь японский флот был создан на верфях ведущих кораблестроительных держав. В основном Англии. Выбирались только лучшие проекты и адаптировались под нужные требования. Ядро флота составляли новейшие корабли, характеристики которых в наибольшей степени подходили именно для той войны и того противника, которые выбрала сама Япония. Зачастую это были сильнейшие в мире корабли того времени. Или же превосходство в боевой мощи было достигнуто за счет отказа от тех характеристик, которыми в сложившихся условиях можно было пренебречь» \url{https://oldadmiral.livejournal.com/37890.html}
\end{textcitation}
пишет Олд-Адмирал.

\begin{figure}[h!tb] 
	\centering\includegraphics[scale=0.2]{Data/RYAV_sily_storon/A7N2WR2L-TY.jpg}
	%	\label{fig:scipion} % Unique label used for referencing the figure in-text\end{document}
	%	%\addcontentsline{toc}{figure}{Figure \ref{fig:placeholder}} % Uncomment to add the figure to the table of contents%----------------------------------------------------------------------------------------
	\caption{Микаса. Фото неизвестного автора. Взято с Википедии \url{https://en.wikipedia.org/wiki/Japanese_battleship_Mikasa}
	}
	%	CHAPTER 2
\end{figure}

Тут легко впасть в конспирологию. Для любителей возводить тезисы про “англичанку” в абсолют замечу, что среди Британского руководства велись ожесточенные споры касательно продажи современных кораблей развивающимся странам. При этом, как подчеркивает Мак Нил, для некоторых верфей заграничные заказы были единственным способом остаться на плаву.

Итак, у нас две державы. Первая пытается создать флот самостоятельно. Другая закупает почти полностью за границей. Преимущество должно перейти к первой, время работает на нее. Но вот времени как раз-таки у России не было. Значит, стратегия Японии оказалась верной: победа, в конечном итоге, осталась за ними. (4)

Еще одним причиной грядущего поражения стала дрянная подготовка экипажей. Учения не проводились, тактику не преподавали, артиллерийские стрельбы велись редко. Корабли стояли в доках, матросов порой бессмысленно гоняли. Все это откликнется в грядущих боях.

Русско-японская война оказывалась противостоянием одной лишь части созданного практически своими силами отечественного флота с флотом японским, сконструированным на Западных Верфях.


\section{Армия и инфраструктура}

Говоря о русско-японской войне надо понимать, что велась она не только за тысячи верст от Санкт-Петербурга, но и за сотни верст от границы. Отдаленность Порта-Артура делала коммуникации, по словам Айрапетова, растянутыми и уязвимыми. Транзитных баз не имелось. Единственный док для крупных судов располагался во Владивостоке (у японской империи таковых было четыре).

Условия проживания в Порт-Артуре были очень суровыми: болезни, недостаток воды, проблемы с питанием. Не лучше ситуация была и в Дальнем (ныне – Далянь)

Китайские укрепления в Порт-Артуре не годились даже для защиты от каких-нибудь древних кочевников. Было начато строительство фортификационных сооружений. К началу войны, пишет Айрапетов, работы закончили только половину. О том, как это сказалось на обороне, будет сказано в следующих статьях.
Русская армия насчитывала около 1350 тысяч человек и еще 3 с половиной миллиона имелось в резерве.

Оценки японской армии расходятся. Изначально РИ оценивала силы противника в 358 тыс. чел., из них 217 тыс. резервистов. Также предполагалось, что корпус Японии на континенте не превысит 250 тысяч человек. Такое недооценивание стало критичным: Империя Восходящего Солнца мобилизовала в дальнейшем 1,1 миллион и перебросило на фронт 500 тысяч.

\begin{figure}[h!tb] 
	\centering\includegraphics[scale=0.4]{Data/RYAV_sily_storon/kGNdyBmjr4c.jpg}
	%	\label{fig:scipion} % Unique label used for referencing the figure in-text\end{document}
	%	%\addcontentsline{toc}{figure}{Figure \ref{fig:placeholder}} % Uncomment to add the figure to the table of contents%----------------------------------------------------------------------------------------
	\caption{Японская пехота во время Боксерского восстания.  \url{https://477768.livejournal.com/2794115.html}
	}
	%	CHAPTER 2
\end{figure}

На начало войны на Дальнем Востоке русская армия составляла 133 000 человек. Главнокомандующий Куропаткин предполагал постепенное отступление вглубь Маньчжурии, накопление сил, благо Транссиб позволял перебрасывать войска, и контрнаступление, десант в Японии и пленение Микадо. По его расчётам для победы требовалось шесть корпусов. При этом конкретного операций у Куропаткина не имелось, только «общие контуры». Плохое управление оказалось вторым критическим недостатком.

Транссибирская магистраль действительно стала грозным оружием, несмотря на все недостатки. Для переброски армейского корпуса требовалось около 60-70 дней (хотя в теории выходило около 45).

Японская война стала вторым конфликтом после русско-турецкой войны 1877—1878 гг. с применением призывной армией и первой с испытанием массовых резервов. Тут всплыла первая неприятность: основным оружием была новая и еще не освоенная винтовка Мосина, а большинство призывников умели обращаться лишь с «Берданками».

Второй ошибкой стало недооценивание пулеметов. «В начале боевых действий на Дальнем Востоке находилась одна пулеметная команда из 8 пулеметов», - пишет Айрапетов. Наверстывать отставание пришлось уже во время войны.

Итак, Российскую империю назвать отсталой нельзя. Тем трагичнее станут такие просчеты, как экономия на учениях, плохое планирование операций. При этом стоит заметить, что время работало на нас и для победы японцам необходимо было действовать в коротком временном окошке.


Автор Виктор Пепелов. Оригинал \url{https://vk.com/wall-162479647_159545} 
К началу \ref{tablecont}

\#Пепел@catx2

\#Заметка@catx2
