\chapter{Так погибли римские легионы}

Общеизвестно что всеми нами любимые римские легионы погибли в Кризисе Третьего Века, просто сгорели в огне бесконечной гражданской войны. И потом Империя уже не пыталась их возрождать, опираясь на ополчения и федератов. Но многим непонятно "зачем они сломали лучшее что было в Империи, и на чем она, собственно, стояла?". Давайте разбираться.


Однажды молодой и не очень умный император Александр Север начал выказывать неуважение свей армии, задерживать жалование и всячески отодвигать солдатню от власти. Не то чтобы это было что-то уж совсем неправильное, но он както резко переобулся в воздухе от любви и обожания своей солдатни, плюс сам по себе подставился, начав инспектировать войска. После чего был закономерно прирезан как свинья, вместе со своей мамашей, а Максимилиан Фракиец, римский полководец, объявил себя императором, и пошел на Рим, взял его и начал править. Недолго, правда, его совсем скоро убьют. Это был первый "солдатский император", но не последний, дальше их будет очень много.


Северы совершили чудовищную по своей тупости херню. Они сначала опирались онли на армию ("плати солдатам и об остальном можешь не думать" — это оттуда), а потом, ВНЕЗАПНО осознали что трон вообщет стоит на штыках, никто их не любит. И, более того, армия осознавала что она вообщет единственная опора трона и Северы от неё зависят. Северы начали резко менять курс и отбирать у легионов их неприлично выросшие рычаги влияния. Армия обиделась и убила Северова потом устроила в стране такой пиздец, что там императоры прост как презервативы менялись, по штуке в год, среднеарифметически. Легионы отторгли государство, полезли в политику и начали ставить своих людей, а всех остальных просто убивать. И это почти на сотню лет.


В итоге армия проиграла, солдатских императоров перебили а легионы разбили бошки друг об друга. Новой власти оно надо вообще, солдатня, лезущая к нам в реп? Не надо. Поэтому начали опираться на ополчение и федератов. Эти в политику вообще не лезли, им допизды. Максимум пограбят когот. Никаких больше походов на Рим, бунтов целыми армиями, претендентов-генералов, тихо-мирно. Поэтому, короче, и сдохли через две сотни лет, но это уже другая история, а в моменте решение было вполне разумным. 

\begin{figure}[h!tb] 
	\centering\includegraphics[scale=0.6]{LegionsDead/1593599741150795821.png}
	%	\label{fig:scipion} % Unique label used for referencing the figure in-text\end{document}
	%	%\addcontentsline{toc}{figure}{Figure \ref{fig:placeholder}} % Uncomment to add the figure to the table of contents%----------------------------------------------------------------------------------------
	%	\caption{Из Вики}%	CHAPTER 2
\end{figure}