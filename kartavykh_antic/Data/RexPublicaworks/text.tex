\chapter{Как работала Римская Республика? }



Народ спрашивал как так получилось, что во время гражданской войны в Риме ещё молодой Гней Помпей поехал к себе домой в Пицен, набрал там себе легионов и пошел ими с марианцами воевать. На этом фоне удобно показать как Римская Республика вообще работала, а то многие просто натягивают современные нам государства на античку, не сильно вникая в нюансы. А это почти всегда ошибка.


Основным актором внутренней политики в Италии были городские муниципалитеты, там как раз основной движ и происходил. Тоесть Помпей не сам по себе легионы нанимал, он это дело спонсировал, а набирались они городами Пицена. Понятно, да? Даже во времена полнейшей анархии какой-то хуй с горы НЕ мог ничего набрать, более-менее толкового, если не имел поддержку среди элиты той области, в которой находился. В случае с Помпеем он не столько даже спонсировал, сколько выступал в роли этакого центра притяжения, вокруг которого собирались крупные и средние магнаты, и башляли бабки. Плюс они же, так как именно из них или из их клиентов, и состояло местное самоуправление, этим набором и занималось. Ну а уже готовую армию возглавил Помпей. Примерно так.


Если рассматривать более стабильную ситуацию, не когда у тебя уже пять лет гражданская война идет, то схема более вменяемая. Формируется госзаказ на армию из, скажем, пяти легионов. Берется не какой-то ноунейм с корочкой от Сената (томушта его просто нахуй пошлют), а один из видных чуваков ИЗ РИМА, желательно со связями в той местности где планируется набор, и он уже берет под козырек, получает бабки (обычно не из казны, а Сенат решает что вон те и вон те скидываются на армию, тоесть это частные капиталы всё, опять таки), после чего этот олигарх пиздует на место сбора, напрягает своих местных корешей и клиентов а также клиентов корешей и корешей клиентов и всех их просит "по братски сделать красиво", что те, собственно, и делают. С одной стороны у него закон на руках, о наборе, это да. Но нельзя забывать что бюрократия в Риме пошла уже в имперское время, это там "без бумажки ты букашка". В Республике, если закон не подкреплен золотом или железом, то хуй на него класть не возбраняется. НО, это очень важно, он почему закон? Томушта в Риме собрались все кто в Республике вообще имеет вес, и которые через вторые-третьи руки имеет в клиентах каждого, кто хоть на что-то влияет. И они, своим ебическим суммарным весом, скрепляют закон. Тоесть если элита сказала что "надо", то система работала практически идеально, ведь у нас каждый магистрат на местах не просто разнарядку из Рима выполняет, а оказывает услугу вполне конкретным политикам, которые, в свою очередь, оказывают услугу более высоким политикам и так далее. Уровень частной инициативы зашкаливал, это было соцсоревноварие "здорового человека". Вот так было когда элита была едина во мнении.


Но так было не всегда, особенно когда популяры с оптиматами начали по серьезке ебшиться а Сенат и Народное Собрание, как два бешоных принтера, штамповали законы потиворечащие друг другу. В этих условиях всякая шизофреническая хуйня случалась регулярно, и тут для того кто закон продавливал было важно не просто его отстоять, но и найти людей которые его могут выполнить, дать им такую возможность, а также уговорить или отпиздить тех, кто этого закона не хочет. Все опять завязано не на бумажки а на живые пирамиды из людей.


Это вообще ключевое, когда мы говорим про античку. Олигархическая республика, чисто мафиозная система лояльности, нанимаемые политиками частные армии, муниципальное самоуправление. Это, короче, этакий античный киберпанк. Ещё следует сказать что весь ВПК был тоже частный, в основном оно пилилось на латифундиях (ниебических размеров колхозы крупнейших римских олигархов) руками рабов, а оставшееся делали города под себя. Город, а точнее община, в данном случае выступал как коллективный олигарх, томушта только так он на что-то влиял в римской политоте.


Напоследок скажем про деньги. Годовой бюджет Республики времен Суллы был примерно в 15 раз меньше суммарного капитала полусотни сильнейших римских родов. Не помню цифр, помню что у одного Красса, на пике (середина шестидесятых, примерно) в личной собственности было имущества и капиталов от трети до половины годового бюджета страны. Тоесть Республика времен расцвета, Восток уже завоеван, десятки миллионов человек, а он ОДИН был сопоставим с ней по финансовым возможностям. Туда же налоговые аукционы. Республика не собирает налоги, Республика продает право собирать налоги частным лицам и они уже едут и на месте рулят движем. То же самое и со всеми другими должностями. Абсолютно нормально было сначала заплатить бешоные бабки на взятки чтобы получить должность, а потом за свои же деньги ремонтировать дороги или храмы строить всякие. Суть в том,что бюджета вообще не было, никто его так не воспринимал. Это была кубышка (общаг, если так понятнее), куда патриции просто скидывали бабки от завоеваний и прочих менее прибыльных занятий, понемногу, а потом совместно решали куда им их потратить.


И государства тоже не было, а была Рес Публика, тоесть Общее Дело. Фактически, корпорация римских нобилей высшего эшелона, в которую на роли младших партнеров были приглашены средний класс и народ, плюс всякие союзники и прочее. Созданная томушта так сподручнее мир захватывать и грабить его.


А Империя это потом.

\begin{figure}[h!tb] 
	\centering\includegraphics[scale=0.6]{RexPublicaworks/1579286098174274121.png}
	%	\label{fig:scipion} % Unique label used for referencing the figure in-text\end{document}
	%	%\addcontentsline{toc}{figure}{Figure \ref{fig:placeholder}} % Uncomment to add the figure to the table of contents%----------------------------------------------------------------------------------------
	%	\caption{Из Вики}%	CHAPTER 2
\end{figure}