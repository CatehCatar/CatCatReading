\chapter{И еще раз про мифы}

Пара слов о самых спорных вопросах.

Самый главный спор всего, что связано с Римон-20 - это количество сбитых самолётов. 4 или 5. С моей точки зрения, спор совершенно бессмысленный, так как не меняет ничего ни в итоговом результате, ни в ходе событий. Но попробуем разобраться.

Картинка ниже - скан лётной книжки Ифтаха Спектора, которому была засчитана половинка той самой "пятой" победы. Самое удивительное, что засчитали ему ёё только в 73-м году (запись розовым), по данным разведки. Что такого накопали АМАН - я не знаю. Наиболее реалистичная гипотиза - в том, что самолёт капитана Макары был списан или канибализирован после того боя.

Пару слов о других мифах.
Миф: Главным фактором поражения стала грамотно организованная засада.

Правда: Засада была организована столь неудачно, что по факту имела скорее психологический эффект. Шли по шерсть - вернулись стриженными. "Фантомы" вместо ракетной атаки с дистанции, влезли в манёвренный бой. Единственный "плюс" - израильтяне столкнулись с 8 самолётами, а не с 18, как могло бы быть. Более того, из 16 запланированных ими к участию самолётов, 2 сломались, один потерялся, трое опоздали, двоих навели не туда. Из опоздавших один умудрился вместе с топливными баками сбросить ракеты (!). Отсюда, кстати, следующий миф.

Миф: в бою участвовало 20 советских самолётов и 16 израильских.

Правда: было 8 на 8. 20 - это с учётом ещё 3 звеньев, не успевших к месту боя. Причем реально успеть шансы были только у звена Саранина. Ещё можно добавить пару Колесов-Пушкарский, взлетевших через 40 минут после начала боя ... с Котмии, отобрав у арабов пару самолётов.
По израильтянам - написано выше. В основной фазе участвовало 8, в конце - 11.

Миф: израильтяне сбежали, не рискнув связываться с подходящими советскими силами.

Правда: они изначально не собирались продолжать, на задачу было назначено только 16 самолётов, 4 из которых по разным причинам вышли из боя раньше, а у 7 уже закончилось топливо и ракеты.